% Indicate the main file. Must go at the beginning of the file.
% !TEX root = ../main.tex

%%%%%%%%%%%%%%%%%%%%%%%%%%%%%%%%%%%%%%%%%%%%%%%%%%%%%%%%%%%%%%%%%%%%%%%%%%%%%%%%
% 02_methods
%%%%%%%%%%%%%%%%%%%%%%%%%%%%%%%%%%%%%%%%%%%%%%%%%%%%%%%%%%%%%%%%%%%%%%%%%%%%%%%%

\section{Methods}
\label{methods}

\subsection{Dataset}

The dataset used in this study is the InsectSet32 dataset \autocite{faissInsectSet32DatasetAutomatic2022}. 
Containing this description:

\begin{displayquote}
This dataset contains recordings of 32 sound producing insect species with a total 335 files and a length of 57 minutes. 
The dataset was compiled for training neural networks to automatically identify insect species while comparing adaptive, 
waveform-based frontends to conventional mel-spectrogram methods for audio feature extraction. 
This work will be submitted for publication in the future and the dataset can be used to replicate the results or for similar research. 
Roughly half of the recordings (147) are of nine species belonging to the order Orthoptera. 
These recordings stem from a dataset that was originally compiled by Baudewijn Odé (unpublished). 
The remaining recordings (188) are of 23 species in the family Cicadidae. 
These recordings were selected from the Global Cicada Sound Collection hosted on Bioacoustica \autocite{bakerBioAcousticaFreeOpen2015}, 
including recordings published in \autocites{bakerGlobalCicadaSound2015}{poppleRevisionMyopsaltaCrucifera2017}.
Many recordings from this collection included speech annotations in the beginning of the recordings, 
therefore the last ten seconds of audio were extracted and used in this dataset. 
All files were manually inspected and files with strong noise interference or with sounds of multiple species were removed. 
Between species, the number of files ranges from four to 22 files and the length from 40 seconds to almost nine minutes of audio material for a single species. 
The files range in length from less than one second to several minutes. 
All original files were available with sample rates of at least 44.1 kHz or higher but were resampled to 44.1 kHz mono WAV files for consistency.
\end{displayquote}

The files are split into training, validation and test sets. And there are two .csv files containing
the labels and the filenames of the recordings.


\subsection{Programming Language and Frameworks}
To build and train the deep learning model, the programming language Python was used.
The Frameworks PyTorch, Lightning are very popular and powerful tools for building deep learning models.

\subsection{Deep Learning Model}
% Describe the final model architecture

\subsection{Data Processing}
A custom Dataloader was implemented, to handle the data processing on the fly
and provide the trainer with then data samples matching the chosen indices.
To implement the dataloader, the PyTorch Dataset and DataLoader classes where used.
For the data processing, the torchaudio and numpy libraries where used.
There is two steps to the data processing: Sampling and Transformation.

\subsubsection{Sampeling}
The audio files are of different lengths and the model can only handle inputs of a fixed size.
Since the smallest files are of a length of around 1 second and the longest file is around
160 seconds, a compromise had to be made. To only sample the files to a length of 1 second
would mean very little information being available for the model to learn from. On the other
hand if the files are sampled for a length of more than a second, the short files would need
to be padded with zeros meaning the file starts with a basically empty part. This could lead
to the model learning from the length of the empty part and not the actual audio signal.
To avoid this, the audio files where sampled to a random length between 1 and 10 seconds and
then padded with zeros to the fixed length of 10 seconds. To implement this, a custom method
was implemented as shown in Listing \ref{lst:sampeling}.

\begin{lstlisting}[
    language=Python, 
    caption={Python code for the sampeling of the filies}, 
    label={lst:sampeling}]
import numpy as np
import torch

def get_random_part_padded(self, waveform: Tensor, samplerate: int) -> Tensor:

    min_len_in_samples = int(self.min_len_in_seconds * samplerate)
    max_len_in_samples = int(self.max_len_in_seconds * samplerate)

    if self.min_len_in_seconds == -1:
        sample_start_index = -max_len_in_samples
        sample_end_index = None

    else:
        part_length = np.random.randint(min_len_in_samples, max_len_in_samples + 1)
        sample_length = waveform.shape[1]
        part_length = min(part_length, sample_length)
        sample_start_index = np.random.randint(0, sample_length - part_length + 1)
        sample_end_index = sample_start_index + part_length

    waveform_part = waveform[:, sample_start_index:sample_end_index]
    actual_part_length = waveform_part.shape[1]
    pad_length = max_len_in_samples - actual_part_length
    waveform_pad = torch.nn.functional.pad(waveform_part, pad=(pad_length, 0, 0, 0))

    return waveform_pad
\end{lstlisting}


\subsubsection{Transformation}
The model is actually one, usually used for image classification and therefore expects
two dimensional input while an audio file has only one dimension. To transform the samples into
into a format, that the model can handle, a short time Fourier transformation  (STFT) was applied.
The Fourier transformation is a mathematical operation that transforms a function of time
into a function of frequency. An other way to describe this is, that a series of spectrograms
for short time slices of the audio signal are created and aligned in a 2D array - basically
a visualization of the audio signal and therefore something a image classification model can handle.
In addition a second version was implemented, where the STFT where transformed additionally.
The two versions are visualized for a random sample with no padding in Figure \ref{fig:transformations}.
The frequency bins were transformed into mel bins, which are a more human like representation
of the frequency content of the audio signal. This transformation is called a mel-spectrogram
and is commonly used in the field of ecoacoustics \autocite[7]{stowellComputationalBioacousticsDeep2022}.
Both versions of the transformation where transformed into decibels and normalized before 
being passed into the model.

\begin{figure}[h!]
    \centering
    \captionsetup{width=.7\linewidth}
    \includegraphics[width=0.85\textwidth]{figures/compare_spectrogram.pdf}
    \caption{Visualization of the two transformations of the audio signal.}
    \label{fig:transformations}
\end{figure}

To implement the two varieties of the transformation, the library torch and torchaudio where used.
A custom transformation method was implemented, that can be used as a layer in the model as shown in 
Listing \ref{lst:transformation}. Some of the parameters of the transformation where made configurable
and others dependant on them where calculated. In the hyperparameter tuning phase, the only parameter
that was tuned was the number of mel bins, which was set to either 64 to try the mel-spectrogram or -1
to use the standard spectrogram.

    \begin{lstlisting}[
        language=Python, 
        caption={Python code for the transformation of the audio signal}, 
        label={lst:transformation}]
    import torch
    import torchaudio

    class NormalizeSpectrogram(torch.nn.Module):
        def forward(self, tensor):
        return (tensor - tensor.min()) / (tensor.max() - tensor.min())

    normalize_transform = NormalizeSpectrogram()

    if n_mels == -1:
        spectogram = torchaudio.transforms.Spectrogram(
            n_fft=n_fft, 
            hop_length=int(n_fft/2), 
            win_length=n_fft)
    else:
        spectogram = torchaudio.transforms.MelSpectrogram(
            n_fft=n_fft,
            hop_length=int(n_fft/2),
            win_length=n_fft,
            n_mels=n_mels,
            f_max=self.sample_rate / 2)

    db_transform = torchaudio.transforms.AmplitudeToDB(top_db=top_db)

    self.transform = torch.nn.Sequential(
        spectogram, 
        db_transform, 
        normalize_transform)
    \end{lstlisting}


\subsection{Fitting the Model}

The training was completely handled by the PyTorch Lightning framework. The logging was done
with the TensorBoard logger and with an additional custom logger to get easier access to the
data afterwards. The hyperparameter grid search was implemented with a custom Python script.

\subsubsection{Training}

The training was done on a single GPU. The model was trained for a maximum of 2000 epochs
with a early stopping callback, that stopped the training if the validation loss did not improve
for patience = 100 epochs. The model was trined with a batch size of 10. The optimizer used was the AdamW
optimizer of the PyTorch library with a weight decay of 0. The loss function used was the
CrossEntropyLoss function of the PyTorch library with default parameters and class weights
according to available data per class. Different learning rates where tried during the hyperparameter
tuning referred to in Section \ref{hyperparameter_tuning}. Only the best model was saved and used for
the evaluation of the model. In order to simplify the evaluation, on completion of the training
the whole dataset was predicted and saved to a csv file in the log folder.

\subsubsection{Hyperparameter Tuning}
\label{hyperparameter_tuning}

For the hyperparameter tuning, a select number of hyperparameters where chosen to be tuned.
Considerations like the experience of some early tests, the computational resources available
and the time frame of the project where taken into account. The hyperparameters that where
chosen for the grid search are shown in Table \ref{tab:hyperparameters}. For the grid search,
models for all possible combinations of the hyperparameters - 
in this case \( 2 \times 3 \times 2 \times 3 = 36 \) where trained. To implement the grid search a short Python
script was written to create the system commands to start the training with the different
hyperparameter combinations.
\begin{table}[h]
    \centering
    \caption{Hyperparameters and values used for the grid search.}
    \label{tab:hyperparameters}
    \begin{tabular}{|l|l|l|l|}
    \hline
    \textbf{Hyperparameter} & \textbf{Description}                  & \textbf{Variations}   & \textbf{Values} \\ \hline
    n\_mels                 & transformation (-1 for regular STFT)  & 2                     & 64, -1 \\ \hline
    n\_res\_blocks          & number of res blocks                  & 3                     & 2, 3, 4 \\ \hline
    learning\_rate          & step size during optimization         & 2                     & 0.001, 0.0001 \\ \hline
    kernel\_size            & dimension of the filter               & 3                     & 3, 5, 7 \\ \hline
    \end{tabular}

\end{table}

