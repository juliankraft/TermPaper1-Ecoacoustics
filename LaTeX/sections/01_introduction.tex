% Indicate the main file. Must go at the beginning of the file.
% !TEX root = ../main.tex

%%%%%%%%%%%%%%%%%%%%%%%%%%%%%%%%%%%%%%%%%%%%%%%%%%%%%%%%%%%%%%%%%%%%%%%%%%%%%%%%
% 01-introduction
%%%%%%%%%%%%%%%%%%%%%%%%%%%%%%%%%%%%%%%%%%%%%%%%%%%%%%%%%%%%%%%%%%%%%%%%%%%%%%%%

\section{Introduction}
\label{section1}

\subsection{Background}%%%%%%%%%%%%%%%%%%%%%%%%%%%%%%%%%%%%%%%%%%%%%%%%%%%%%%%%%%

One of the global environmental issues is the loss of biodiversity \autocite{cardinaleBiodiversityLossIts2012}. 
The term biodiversity is a contraction of biological diversity, which refers to the variety and variability of life forms on Earth.
In recent years, a massive decline in biodiversity has been observed, which is mainly due to human activities. 
The loss of biodiversity is a major concern because it can have a significant impact on the ecosystem and the services it provides \autocite{brondizioGlobalAssessmentReport2019}. 
In order to quantify biodiversity and monitor its changes, it is essential to have a reliable and efficient method for quantifying biodiversity.
Traditional methods for quantifying biodiversity are time-consuming and expensive, and they are not suitable for large-scale monitoring.
But what if there was a non invasive method that could be used to monitor biodiversity in a fast and efficient way?

Bioacoustics, the study of sound production, dispersion and reception in animals, is an interesting
field of research that focusses on the side of an individual species, group or community
and their interaction with each other or their environment. 
It has provided amazing insights into the behavior and communication of animals.
Ecoacoustics, a field of research very close to bioacoustics has a slightly different focus.
it is the study of the soundscape of an ecosystem - the sounds produced by all living organisms in an ecosystem.
The questions are typically a bit broader and about the composition of the soundscape, the temporal and spatial patterns of sound production, 
and the relationship between the soundscape and the environment. The study of biodiversity using bioacoustics
is one of its promising applications even though there are still some issues to be fixed \autocite{scarpelliMultiIndexEcoacousticsAnalysis2021}. 
These passive acoustic monitoring (PAM) like the using of sound recordings to monitor biodiversity are currently widely researched
and developed and even combined with modern artificial intelligence (AI) methods \autocite{dengHarnessingPowerSound2023}.
Before there can be a holistic approach to the monitoring of biodiversity using ecoacoustics, some smaller steps have to be taken.
One of these steps is the classification of the sounds of individual species or taxonomic groups.
So far the taxonomic coverage includes birds, Cetaceans and other marine mammals, bats, terrestrial mammals,
anurans, insects and fish \autocite[4]{stowellComputationalBioacousticsDeep2022}.
Birds are the most studied group of animals in the field of bioacoustics, with a
commonly known and distributed application called the "BirdNET" \autocite{kahlBirdNETDeepLearning2021}.

The focus of this study is to reproduce the results of the paper \autocite{faissInsectSet32DatasetAutomatic2022} and to create a model 
that can classify a subset of insects using their voices. The subset contains members of the two groups of insects: \textit{Orthoptera} and \textit{Cicadidae}.
\textit{Orthoptera} is an order of insects that includes grasshoppers, crickets, and katydids. \autocite{capineraOrthoptera2008}
\textit{Cicadidae}, a members of the superfamily Cicadoidea Westwood are four-winged insects with sucking 
mouthparts that possess three ocelli and a rostrum that arises from the base of the head. \autocite{sanbornCicadasHemipteraCicadoidea2008}
The goal is to proof that this technology is accessible for everyone with the knowledge and a 
regular gaming computer with a graphic processing unit (GPU). To achieve this, a model will be
constructed and trained using the same dataset as the original paper. A hyperparameter tuning will be performed to find the best
configuration for the model. The results of the hyperparameter tuning will be evaluated and discussed.
The best performing model will be tested and evaluated for its performance and accuracy. The results will be
discussed and compared to the results of the original paper.
