% Indicate the main file. Must go at the beginning of the file.
% !TEX root = ../main.tex

%%%%%%%%%%%%%%%%%%%%%%%%%%%%%%%%%%%%%%%%%%%%%%%%%%%%%%%%%%%%%%%%%%%%%%%%%%%%%%%%
% 01-introduction
%%%%%%%%%%%%%%%%%%%%%%%%%%%%%%%%%%%%%%%%%%%%%%%%%%%%%%%%%%%%%%%%%%%%%%%%%%%%%%%%

\section{Introduction}
\label{section1}

\subsection{Background}%%%%%%%%%%%%%%%%%%%%%%%%%%%%%%%%%%%%%%%%%%%%%%%%%%%%%%%%%%

The question about biodiversity and its importance has been a topic of interest for many years. 
The term biodiversity is a contraction of biological diversity, which refers to the variety and variability of life forms on Earth.
In recent years, a massive decline in biodiversity has been observed, which is mainly due to human activities. 
The loss of biodiversity is a major concern because it can have a significant impact on the ecosystem and the services it provides \autocite{brondizioGlobalAssessmentReport2019}. 
In order to quantify biodiversity and monitor its changes, it is essential to have a reliable and efficient method for measuring biodiversity.
Traditional methods for measuring biodiversity are time-consuming and expensive, and they are not suitable for large-scale monitoring.
But what if there was a non invasive method that could be used to monitor biodiversity in a fast and efficient way?
Ecoacoustics might deliver a promising solution to this problem even tough there remain some issues to be fixed \autocite{scarpelliMultiIndexEcoacousticsAnalysis2021}.  
Passive acoustic monitoring (PAM) like the using of sound recordings to monitor biodiversity are currently widely researched
and developed and even combined with modern artificial intelligence (AI) methods \autocite{dengHarnessingPowerSound2023}.

% discuss some examples

The focus of this study is to reproduce the results of the paper \autocite{faissInsectSet32DatasetAutomatic2022} and to create a model 
that can classify the insect sounds with a high accuracy. Furthermore the model will be tested and evaluated for its performance and
accuracy. The results will be discussed and compared to the results of the original paper. The goal is to prof that this technology
could be accessible for everyone with the knowledge and a regular gaming computer with a graphic processing unit (GPU).

\subsection{Insects of Interest}%%%%%%%%%%%%%%%%%%%%%%%%%%%%%%%%%%%%%%%%%%%%%%%%%%

There are countless species of insects in the world, and many of them produce sounds for various reasons.
In this study, we are interested in the sounds produced by two groups of insects: Orthoptera and Cicadidae.
Orthoptera is an order of insects that includes grasshoppers, crickets, and katydids. \autocite{capineraOrthoptera2008}
Cicadidae, a members of the superfamily Cicadoidea Westwood are four-winged insects with sucking 
mouthparts that possess three ocelli and a rostrum that arises from the base of the head. \autocite{sanbornCicadasHemipteraCicadoidea2008}




