% Indicate the main file. Must go at the beginning of the file.
% !TEX root = ../main.tex

%%%%%%%%%%%%%%%%%%%%%%%%%%%%%%%%%%%%%%%%%%%%%%%%%%%%%%%%%%%%%%%%%%%%%%%%%%%%%%%%
% Abstract
%%%%%%%%%%%%%%%%%%%%%%%%%%%%%%%%%%%%%%%%%%%%%%%%%%%%%%%%%%%%%%%%%%%%%%%%%%%%%%%%

\vspace*{\fill}

\section*{Abstract}
\label{abstract}

Biodiversity is rapidly declining worldwide. The first step towards understanding the causes and 
taking measures to counteract this trend is the development of methods and tools to quantify and monitor it. 
The biodiversity of insects is highly relevant due to their crucial role in ecosystems.
Insects contribute to processes such as pollination, decomposition, and serving as a food source for other species.
While many methods exist to monitor insects, ecoacoustics provides a non-invasive paradigm to measure species via acoustic signals.
Recently, data-driven approaches, particularly deep learning-based methods, have proven effective in detecting insect species from acoustic in situ field measurements.
These advancements offer promising new ways to understand, monitor, and protect insect populations.
This study aims to reproduce the results of a previous research on insect sound 
classification using deep learning and to assess the accessibility of this technology. 
Using the InsectSet32 dataset, which contains 335 recordings of 32 insect species, a 
convolutional neural network (CNN) with residual blocks (ResNet) was developed. 
The model was implemented and trained using the 
PyTorch framework and optimized via hyperparameter tuning. The best-performing model 
achieved an accuracy of 0.706 on the validation set and 0.649 on the test set, 
closely aligning with the original study's findings. This work demonstrates that, 
with appropriate knowledge and relatively affordable hardware, such as a regular 
gaming computer equipped with a graphics processing unit (GPU), non-experts can effectively utilize deep 
learning models for ecoacoustics applications. The study underscores the potential 
of combining ecoacoustics with artificial intelligence to advance biodiversity monitoring, 
providing a non-invasive and efficient method for large-scale environmental assessments.

\vspace*{\fill}