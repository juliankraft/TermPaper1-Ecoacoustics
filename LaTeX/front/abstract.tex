% Indicate the main file. Must go at the beginning of the file.
% !TEX root = ../main.tex

%%%%%%%%%%%%%%%%%%%%%%%%%%%%%%%%%%%%%%%%%%%%%%%%%%%%%%%%%%%%%%%%%%%%%%%%%%%%%%%%
% Abstract
%%%%%%%%%%%%%%%%%%%%%%%%%%%%%%%%%%%%%%%%%%%%%%%%%%%%%%%%%%%%%%%%%%%%%%%%%%%%%%%%

\vspace*{\fill}

\section*{Abstract}
\label{abstract}

This study aims to reproduce the results of a previous research on insect sound 
classification using deep learning and to assess the accessibility of this technology. 
Using the InsectSet32 dataset, which contains 335 recordings of 32 insect species, a 
convolutional neural network (CNN) with residual blocks was developed. The model was trained using the 
PyTorch framework and optimized through hyperparameter tuning. The best-performing model 
achieved an accuracy of 0.706 on the validation set and 0.649 on the test set, 
closely aligning with the original study's findings. This work demonstrates that, 
with appropriate knowledge and relatively affordable hardware, such as a regular 
gaming computer equipped with a GPU, non-experts can effectively utilize deep 
learning models for ecoacoustic applications. The study underscores the potential 
of combining ecoacoustics with artificial intelligence to advance biodiversity monitoring, 
providing a non-invasive and efficient method for large-scale environmental assessments.

\vspace*{\fill}