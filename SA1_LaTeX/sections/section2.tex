% Indicate the main file. Must go at the beginning of the file.
% !TEX root = ../main.tex

%%%%%%%%%%%%%%%%%%%%%%%%%%%%%%%%%%%%%%%%%%%%%%%%%%%%%%%%%%%%%%%%%%%%%%%%%%%%%%%%
% SECTION 2
%%%%%%%%%%%%%%%%%%%%%%%%%%%%%%%%%%%%%%%%%%%%%%%%%%%%%%%%%%%%%%%%%%%%%%%%%%%%%%%%

\section{Materials and Methods}
\label{section2}

\subsection{Programming Language and Frameworks}
To build and train the deep learning model, the programming language Python was used.
The Frameworks PyTorch, Lightning are very popular and powerful tools for building deep learning models.

\subsection{Data Processing}
To load and process the the data on the fly, a custom data loader was implemented. 
The data loader reads the audio files and their corresponding labels from the dataset 
and applies the necessary transformations to the audio files.

\subsubsection{Sample Size}
The audio files are of different lengths. In order to avoid the model being biased towards 
the length of the audio files, the audio files are sampled to a fixed length. Since there is
files below the fixed length, the audio files are padded with zeros, it seems not enough to 
just sample the files and pad them if needed. This would allow bias to be introduced because
of the padding. To avoid this, the audio files are sampled to a random length between 1 and 5
seconds and then padded with zeros to the fixed length of 5 seconds.

\subsubsection{Transformation}
Before the audio files are fed into the model, they are transformed into a mel-spectrogram -
short for melody spectrogram. A mel-spectrogram is a visual representation of the audio signal 
aiming to mimic the human perception of sound and is commonly used in audio processing tasks like
speech recognition and music genre classification. It does however provide certain advantages
for audio classification in general and can therefore be used in the field of ecoacoustics as well \autocite[7]{stowellComputationalBioacousticsDeep2022}.
The mel-spectrogram is a 2D array that represents the frequency content of the audio signal over time.




\subsubsection{Deep Learning Model}
% Describe the final model architecture


\subsubsection{Training}
% Describe the training process


\subsubsection{Evaluation}
% Describe the evaluation process


\subsection{Hyperparameter Tuning}