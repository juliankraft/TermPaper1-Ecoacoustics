% Indicate the main file. Must go at the beginning of the file.
% !TEX root = ../main.tex

%%%%%%%%%%%%%%%%%%%%%%%%%%%%%%%%%%%%%%%%%%%%%%%%%%%%%%%%%%%%%%%%%%%%%%%%%%%%%%%%
% SECTION 1
%%%%%%%%%%%%%%%%%%%%%%%%%%%%%%%%%%%%%%%%%%%%%%%%%%%%%%%%%%%%%%%%%%%%%%%%%%%%%%%%

\section{Introduction}
\label{section1}

\subsection{Background}

The question about biodiversity and its importance has been a topic of interest for many years. 
The term biodiversity is a contraction of biological diversity, which refers to the variety and variability of life forms on Earth.
In recent years, a massive decline in biodiversity has been observed, which is mainly due to human activities.
The loss of biodiversity is a major concern because it can have a significant impact on the ecosystem and the services it provides.
In order to quantify biodiversity and monitor its changes, it is essential to have a reliable and efficient method for measuring biodiversity.
Traditional methods for measuring biodiversity are time-consuming and expensive, and they are not suitable for large-scale monitoring.
But what if there was a non invasive method that could be used to monitor biodiversity in a fast and efficient way?
Ecoacoustics might just be a possible solution to this problem. Recent developments in the field of ecoaucustics have
shown that it is possible to use sound recordings to monitor biodiversity in a non-invasive way. This method is known as 
acoustic biodiversity monitoring. It is very cost effective since it does not need experts to actually go to the field and
find and document the species. Instead a regular microphone can be used to record the sounds of the environment and theoretically
a regular computer might just provide enough calculating power to analyze the recordings and provide a report on the biodiversity.
In this paper an attempt in this direction is made. The goal is to use an existing dataset of insect sounds and try to create a model
in order to classify the sounds and identify the species.

\subsection{Dataset}

The dataset used in this study is the InsectSet32 \autocite{faissInsectSet32DatasetAutomatic2022} 
This dataset contains recordings of 32 sound producing insect species with a total 335 files and a length of 57 minutes.


\subsection{Research Question}


